% Archivo de ejemplo para colocar introducción
\chapter{Introducción}
\label{cha:intro}
\todo[inline]{\href{https://es.overleaf.com/learn/latex/Learn\_LaTeX\_in\_30\_minutes}{Learn Latex in 30 minutes}}

\todo{escriba en 3ra persona}

Esta sección debe tener a o menos 2 páginas  y un máximo de 3.

\begin{itemize}
	\item La Introducción es la presentación de una pregunta.
	\item ¿Por qué se ha hecho este trabajo? 
	\item  El interés que tiene en el contexto científico.
	\item  Trabajos previos sobre el tema y qué aspectos no dejan claros, que constituyen el objeto de nuestra investigación.
	\item El penúltimo párrafo de la introducción se utiliza para resumir el objetivo del estudio.
	\item El último párrafo se usa para describir la estructura del documento y lo que se encuentra en cada sección.
\end{itemize}

Dar contexto global al trabajo a realizar, en particular debe ser escrito desde lo no particular, tal que interese al lector. Para efectuar esta sección debe leer o considerar a lo menos 20 artículos indexados (se refiere a artículo publicados en revistas académicas de corriente principal, o congresos). Visite por ejemplo:
\begin{itemize}
	\item \url{www.scielo.org}.
	\item \url{http://www.sciencedirect.com/}.
	\item \url{http://biblioteca.unab.cl/} (debe solicitar su \textit{password} a su coordinador).
\end{itemize}

\todo[inline, color=blue]{Más cómodo en español que en inglés --- solo SoS revíselo en google chrome y active traducción a español. Recuerde que google translate permite subir el pdf del paper también.}

Por favor, considere con ``temor y temblor'' a lo largo de todo el texto las siguientes premisas:
\begin{itemize}
	\item Conecte los párrafos y la secciones a través de las ideas que desarrolla.
	\item Respalde sus afirmaciones con citas. Por ejemplo, si usted dice que el 60 \% de una población es pobre, indique el estudio que respalda dicha afirmación.
	\item Evite redundancia.
	\item No más de dos o tres ideas por párrafo.
	\item Sea asertivo y preciso a la hora de escribir (no sobrevuele, ¡aterrice la idea!).
	\item Debe existir un hilo conductor de ideas claramente definido, no puede saltar de ideas en forma arbitraria.
	\item Todo acrónimo o sigla utilizada debe explicarse la primera vez. Por ejemplo si deseo utilizar la sigla \textsc{``EPTE''}, deberé indicar la primera vez que aparece: Estudiantes de Postgrado Tratando de Escribir (\textsc{``EPTE''}). Luego podrá utilizar la sigla sin mayor mención a su significado. Siempre prefiera versalita para las siglas.
	\item No debe redactar de manera coloquial. Esta debe permanecer alejada del objeto de estudio.
	\item Cuide la concordancia (un concepto debe ser esencialmente el mismo a lo largo de todo el manuscrito).
	\item Cuide los tiempos verbales.
	\item Su trabajo debe ser propio, es decir, debe estar ``limpiamente'' escrito y, además, debe ser suyo. \todo[inline]{Busque referencia de paper: nombre de este y la palabra Bib\TeX ---copie la fuente en formato Bib\TeX y cópielo en su archivo \texttt{.bib}. En la mayoría de los indexadores, es seleccionar la opción ``cite this'' o íconos con carácteres como comillas, entre otros. También los puede construir Ud. Adicionalmente, si bien Bib\TeX sigue siendo utilizado, prefiera su evolución, Bib\LaTeX, que soporta mejor escritura no inglesa y causa menos dolores de cabeza. \href{https://www.overleaf.com/learn/latex/Articles/Getting_started_with_BibLaTeX}{Aquí} hay una introducción.}
\end{itemize}
